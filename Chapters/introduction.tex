\chapter{Introduction}
The emergence of \ac{MOOCs} has spawned a sharp growth in the e-learning industry in recent years \cite{_e-learning_2014}. In 2012, as much as one third of all enrollments in the US were registrations for online courses \cite{allen_grade_2014}. In addition, 77\% of US companies offer advanced professional training based on e-learning to their employees \cite{vernau_corporate_2014}. But it is not just professional or academic education that has driven the growth of e-learning. More and more people are turning to online courses for pursuing personal interests as well. One area that is particularly popular in this regard is online language learning \cite{blake_current_2011,_its_2014}.
% write about how many people use online language learning, how big is the market

% source for "pursuing personal interests"?

%This research project looks at one of these tools and how to improve it, which is used by the European market leader in online language learning, Babbel \cite{_learning_????}.

The increasing popularity of online learning has also increased the demands on content creators and their tools. The systems that have been quietly powering most e-learning platforms are mostly invisible to its end-users. They are often referred to as \textit{learning management systems} or \textit{authoring systems} or \textit{tools} and have vastly simplified the creation, management and distribution of educational material. Some systems even track the learning progress and gather detailed usage statistics thereby allowing educators to optimize the learning content iteratively. But, with the complexity of learning content and number of contributors increasing quickly, these systems also run into limits. Most of these systems were not built to be used by a large number of contributors at the same time. Furthermore, most lack the ability of keeping track of changes made to learning content or to quickly revert changes. These problems together with a growing user base and increasing expectations on e-learning platforms necessitate the development of a new kind of authoring system more sophisticated than its predecessors.

This thesis investigates how language learning content could be created and managed more easily by applying a proven paradigm from the realm of software development, called version control, to a \ac{CAT} used at one of the European market leaders in the space of online language learning, called Babbel. Special emphasis is put on fostering collaboration between different professionals and keeping track of an ever-changing and expanding course structure. The goal is to exploit version control's benefits and make them work for a non-technical user group.

\section{Babbel}
Babbel is a company offering interactive language courses online. It was founded in 2007 and employs about 350 people as of Fall 2015. Babbel users can choose one of 14 different languages to learn. Each course is created uniquely for one language combination, which consists of a user's native language, called \ac{DL}, and the language he or she wants to learn, called \ac{LL}. Education experts, linguists and language teachers working for Babbel are responsible for managing and improving existing courses as well as creating new ones. All courses can be accessed on the desktop as well as on iOS and Android devices. A speech recognition system helps users to practice their pronunciation. The service is subscription-based and customers can choose among 1- to 12-month payment plans\footnote{http://www.babbel.com/prices}.

The research described in this thesis was conducted over the course of almost one year at the Babbel headquarters in Berlin. The author was part of a four-member team and responsible for user research and interaction design. The team was created as part of an effort to completely rethink the way language lessons were authored and maintained at Babbel. One particular aspect of this process, the collaboration between authors, translators and quality assurance professionals, was the main research theme of this thesis, which is described in more detail below.

\section{Motivation} % problem description
With the accelerating adoption of e-learning and rising demands in regards to the user experience of e-learning platforms, the pressure on content creators to deliver high-quality learning content is increasing steadily. But, in order to deliver content that meets these expectations, authoring tools are needed that facilitate collaboration and quality assurance. It seems that many authoring tools were built under the assumption that creating new learning content is a mostly solitary task. But today, especially at Babbel, this is no longer the case. To create a new language course a number of experts are needed. Until a new course is released, there are at least 5-6 individuals who have participated in the production of the course, often even more. The authoring tool currently used at Babbel does not facilitate this workflow very well and has a couple of inherent problems:

% above: source for 'built under the assumption'?

\begin{itemize}
  \item \textit{Collaboration is inconvenient:} Concurrent editing is almost impossible and content needs to be partially locked so that contributors do not interfere with each other.
  \item \textit{Reverting changes is difficult:} A change, after it was saved, cannot be easily reverted. Instead it needs to be undone manually.
  \item \textit{Changes are not tracked:} The system is more or less blind to what was changed, by whom and when. There is no history, which would allow inspecting past changes.
  \item \textit{Quality assurance:} Because changes are not tracked, \ac{QA} professionals need to do more work than necessary, just because they do not know what has changed exactly. Sometimes new content is also overlooked because of that.
\end{itemize}


\section{Research Questions} % Version Control/Proposed Solution/
The problems described above have one commonality: they are not exclusive to content authoring, but frequently occur during large software development projects as well. Most developers mitigate these problems by using a \ac{VCS}. Initially, version control adds a small overhead to the software development workflow, but this usually pays off when projects become more complex \cite{spinellis_version_2005}.

Despite its obvious benefits, version control has not been widely adopted outside of the programming realm. This could be partially due to its usability issues \cite{church_case_2014,perez_de_rosso_whats_2013}.
In the context of this thesis it was investigated if and how authoring tools could benefit from incorporating version control and how such a system could be made more user-friendly. The research question, as already defined in the research proposal \cite{kreiser_master_2015}, reads as follows:

\begin{center}
% original research question from the proposal
%\textit{What is the best way of exposing version control features to a non-technical user-group in the context of a content authoring tool?}
% Can version control be integrated into authoring tools
 \textit{How can version control improve the authoring process for e-learning content?}
\end{center}

\noindent %The next paragraph is not indented
The research question can be further divided into several sub-questions:

\begin{enumerate}
  \item Which features can be simplified or omitted while  preserving the usefulness of version control?
  %\item How can the interaction design address the diverse needs of different users?
  %\item How can a compromise be found between efficiency and learnability?
  %\item How can the weak learnability of VCSs be improved in order to make the interaction more pleasing for inexperienced users?
  \item How can the learnability of VCSs be improved in order to make them more attractive to a non-technical audience?
  \item How can the overhead, usually introduced through a version control system, be reduced?
\end{enumerate}

\section{Methodology}
In order to answer the research question stated above a rather practical approach was adopted. The interface for a new content authoring tool was conceived and turned into prototypes, which were then tested with real users. Emphasis was laid on the version control features as well as the usability of the overall system. In general, the research project consisted of two major phases: the analysis phase and the design and prototyping phase.

During the analysis phase a literature review was conducted as well as two additional analyses performed. The literature review examined research concerned with the usability of popular version control software as well as examples where version control was applied outside of the programming realm. The subsequent analyses were mainly centered around the future users of the system and their goals. Therefore, a task analysis and a user requirements analysis were performed. These ensured that user needs and goals were properly addressed during the subsequent design phase.

Based on the insights gained during the first phase a prototype was designed that incorporated the most important version control features needed for content authoring. The prototype was semi-interactive and tested with 5 users during a task-based usability study. This first study was followed by a focus group meeting during which users discussed the problems they encountered. The focus group put special emphasis on the version control terminology since finding alternatives to the technical and abstract terms used in most version control systems was a key factor for improving the usability of the interface.

The results of the focus group and the first usability study were then used to design an improved version of the interface, which was tested again during a final usability study. The findings informed the final design of the authoring tool and additionally helped to answer the research question described above.


\section{Thesis Structure}
The next chapter (Chapter \ref{chapter:version-control}) will introduce the reader to version control software in more detail. Afterwards, Chapter \ref{chapter:related-work}, introduces the reader to relevant research in adjacent fields. This is followed by a more in-depth description of the user groups and the tasks that constitute the content authoring process in Chapter \ref{chapter:user-research}.
% maybe the user groups need to be further specified? first time they are mentioned in the thesis

In the second half of the thesis the design process as well as the usability tests are covered. Chapter \ref{chapter:first-iteration} is about the first prototype and the first round of user tests. This is followed by a description of the focus group (Chapter \ref{chapter:focus-group}) and the second design iteration (Chapter \ref{chapter:second-iteration}).

Readers who are mostly interested in the outcome of the design phase and the research results can refer to Chapters \ref{chapter:final-design} and \ref{chapter:results-conclusion}. In the last chapter (Chapter \ref{chapter:future-work}) recommendations and suggestions for future research are made.



% long-term goals?

% content at babbel is complex due to 14 times 7 unique combinations
% on average each language offers 100 courses, and more than 1000 lesson (check numbers again)
% tool needs to be powerful, current tool is highly error-prone
% content creation has many similarities to software development (many collaborators, need for high-quality , peer reviews, continously changing content)




%collaboratively by a number of professionals producing the learning material.





%The growth in e-learning is made possible by learning management systems (LMSs) that have automated or vastly simplified the creation of educational material. With the growing user base and increasing demands on e-learning these systems also need to become better.







%This thesis will look at the systems that power the growth of e-learning, almost unnoticed, by most of its users.
%be concerned with the systems behind many e-learning services. Depending on their purpose they are referred to as learning management systems,  or authoring systems.






%not purely professional, many people also learn out of a personal interest in a particular subject matter (citation).




%This is not just done for professional reasons, but also out of personal interest (citation).


%Subjects, that traditionally have been taught by universities only can now be studied by everyone in a self-paced manner. The surge in e-learning offerings has led to a significant reduction in time and resources needed to acquire new skills.



%One area where this is especially true is learning new languages.



% 1. paragraph: growth of e-learning in recent years
% 2. paragraph: authoring systems
% 3. paragraph: thesis topic
% 4. paragraph:









%Ever more people are drawn to self-paced learning online, including a wide variety of subjects.











%Today, more data is generated than ever before in human history \cite{gantz_digital_2012}. The total volume is expected to double about every two years from now. This will lead to a 50-fold increase in data by 2020 compared to 2010. At the same time only 1\% of this huge pool of data is analyzed. This points to a potential problem: Our tools are failing to keep up with the sheer volume of data.










%During the last two years more data has been generated than in the entire human history prior to that period \cite{_big_????}. Computers and the Internet in particular have made adding and storing information vastly more simple.

%This thesis will look at a very small piece of this data and how it is created and managed at a small startup in Berlin, called Babbel.

%The data Babbel cares about is

%Data itself is meaningless. Especially

%the thesis is about language and data/computers.


%This thesis is about two of the most important human inventions: languages and computers.


%Companies, such as Google or Facebook have built empires based on this data.

%The following thesis is about a very small piece of this data.

%But despite this enormous shift the way data is created and managed has not changed that much.

%This thesis is about a particular kind of data: language.

%To write an ‘eye-catching’ opening sentence that will keep the reader’s attention focused;

%the theme is collaboration on data creation, history of data, compraring data


%Move 1: Establishing a research territory
%- by showing that the general research area is important, central, interesting, problematic, etc. (optional)

%- by introducing and reviewing items of previous research in the area (obligatory)


%Move 2: Establishing a niche
%- by indicating a gap in the previous research or by extending previous knowledge in some way (obligatory)

%Move 3: Occupying the niche
%- by outlining purposes or stating the nature of the present research (obligatory)

%- by listing research questions of hypotheses

%- by announcing principal findings

%- by stating the value of the previous research

%- by indicating the structure of the research paper


%One of the key aspects of writing an introduction, in many disciplines, is to attract the interest of the reader


%To ensure that there is a direct relationship between the introduction and the remainder of the dissertation;
%To ensure that you do not promise what cannot be fulfilled or what goes beyond what can reasonably be expected.
