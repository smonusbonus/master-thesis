\chapter{Future Work} \label{chapter:future-work}
As mentioned in the previous chapter, the research mostly focused on determining how an easy-to-use interface for version control could be integrated into a content authoring tool. The long-term implications of utilizing such a system were not studied. A possible follow-up study could investigate whether the long-term benefits of the system really outweigh the short-term increase in labour. Furthermore, it should be confirmed which features of the system are actually used and in which way. For this purpose a Web Analytics tool could be used, which would provide more quantitative data and enable a more detailed insight into the usage patterns of the system.

A second area of future research could explore whether the findings of this project also apply to a wider range of tools than just content authoring for e-learning. Given the increasing amount of content created and published online, especially by amateurs, it is conceivable that general purpose systems, such as Wordpress \cite{_wordpress_????} or Drupal \cite{_drupal_????} could benefit from this as well. In order to find out, more research would be needed in regards to the ability of occasional users to learn and remember such a system.

Summing up, one could say, that the expansion of version control outside of software development is a recent development, which has slowly shifted the focus to usability and simpler interfaces. Probably, there are more use cases for these systems than the current research landscape suggests. If the HCI community continues to prove, that version control is also viable outside the technical domain, we might increasingly see it appear inside consumer products as well.




% 1. How could the presentated research be improved if it were repeated?

%make long-term study and compare to existing/old content authoring tool to see how much it is used and whether it is appreciated

% 2. What would be the appropriate question(s) for future research starting from what is presented here?


% https://guidetogradschoolsurvival.wordpress.com/2011/04/15/how-to-write-future-workconclusions-2/

% What do you think are the next steps to take?
% What other questions do your results raise?
% Do you think certain paths seem to be more promising than others?

% Another way to look at the future work section, is a way to sort of “claim” an area of research. This is not to say that others can’t research the same things, but if your paper gets published, it’s out there that you had the idea. 

% it has been proven that version control can be a useful addition to tools not related to programming.
% a first step has been made in bringing version control to a wider audience. the thesis has shown that VCSs can be useful for content authoring, but it is easy to imagine that content management systems could benefit from this as well.

% merge conflicts were not covered - no solution yet

% Questions to ask:
% 1. How could the presentated research be improved if it were repeated?
% 2. What would be the appropriate question(s) for future research starting from what is presented here?