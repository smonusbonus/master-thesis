\chapter{Summary}
\pagestyle{headings}

The following thesis investigated the possible application of a proven paradigm from the realm of software development, called version control, to the process of authoring e-learning content. Version control is especially useful in settings of ever-changing text-based data that is created by a large number of people. Therefore, it fits the context of content authoring quite well. 

But version control is notoriously hard to use, even for tech-savvy people. For this reason, this project's goal was to develop a simple user interface for version control which could be employed inside a content authoring tool, that is used by non-technical audience.

The interface design was based on task and requirements analysis as well as insights gained from previous research on the usability of version control. During the course of the project the interface was tested twice in two separate usability studies. The results informed the final design proposal presented at the end of the thesis.

The research has shown that version control can be vastly simplified and thus benefit a non-technical user-base as well. The project's intention is to kick off a discussion on the viability of version control outside the realm of programming and make those features available to a wider audience.

%Topic/Research Question:
%The following thesis investigated how version control, a paradigm known from software development, could improve the creation and management of e-learning content. The research was conducted in cooperation with Babbel, one of the European market leaders in the space of online language learning. The company offers 14 different learn languages, each of them consisting of hundreds of unique lessons. These lessons are created and managed by a group of linguists utilizing a custom content authoring tool. The research question asked whether such an authoring tool could benefit from an integrated version control system. 

%The amount of data on the web is rising daily, but the tools used for managing this data have not evolved to meet the increasing requirements. Version control, which is used for large software development projects, is built for ever-changing data and easy collaboration. But the complexity and poor usability of most version control systems have prevented a wider adoption. This project tried to integrate version control and content authoring while laying special focus on usability to make the system attractive to a non-technical audience.

%Methods:
%The design of the system was based on a user and task analysis and was improved iteratively over the course of two usability studies.

%Results/Conclusion:

% Babbel.com, the interactive language learning service, uses a self-made content authoring tool for managing large amounts of data. Creeping featurism has lead to a situation in which usability is at a low point and errors by the user are penalized hard. This project investigates a possible solution to the problem by taking a proven solution from the development world, namely version control, and adapting it for the use by non-technical users through a graphical user interface. The goals being a greater error-tolerance and simplified collaboration between employees.

%The summary consists of one or two pages, and covers all chapters of the report except the appendices. In other words: it should at least summarize the back\-ground/moti\-vation of the work, the goal or research question, the approach, the results, and the main conclusions and recommendations. Divide the summary in paragraphs that roughly correspond to the chapters in the main body of your report. You do not need to explicitly mention the different chapters though.

%Introduce acronyms in your summary \emph{only} if you use them later in the summary. Do not use the \texttt{$\backslash$ac} command in the summary.

%It is not very common to cite references in the summary. It is possible though, particularly if the presented work really builds up on one or more particular papers/reports.

% What is your topic?
% Why is it important?
% What did you do?
% What are your results?

